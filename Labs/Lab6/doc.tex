% PLANTILLA APA7
% Creado por: Isaac Palma Medina
% Última actualización: 25/07/2021
% @COPYLEFT

% Fuentes consultadas (todos los derechos reservados):  
% Normas APA. (2019). Guía Normas APA. https://normas-apa.org/wp-content/uploads/Guia-Normas-APA-7ma-edicion.pdf
% Tecnológico de Costa Rica [Richmond]. (2020, 16 abril). LaTeX desde cero con Overleaf (1 de 3) [Vídeo]. YouTube. https://www.youtube.com/watch?v=kM1KvHVuaTY Weiss, D. (2021). 
% Formatting documents in APA style (7th Edition) with the apa7 LATEX class. https://ctan.math.washington.edu/tex-archive/macros/latex/contrib/apa7/apa7.pdf @COPYLEFT

%+-+-+-+-++-+-+-+-+-+-+-+-+-++-+-+-+-+-+-+-+-+-+-+-+-+-+-+-+-+-++-+-+-+-+-+-+-+-+-+

% Preámbulo
\documentclass[stu, 12pt, letterpaper, donotrepeattitle, floatsintext, natbib, helv]{apa7}
\usepackage[utf8]{inputenc}
\usepackage{comment}
\usepackage{marvosym}
\usepackage{graphicx}
\usepackage{float}
\usepackage[normalem]{ulem}
\usepackage[spanish]{babel} 
\usepackage{gensymb}
%\usepackage{titling}
\let\apasubparagraph\subparagraph
\let\subparagraph\paragraph
\usepackage[compact]{titlesec}
\let\subparagraph\apasubparagraph
\usepackage{hyperref}
\selectlanguage{spanish}
\useunder{\uline}{\ul}{}
\newcommand{\myparagraph}[1]{\paragraph{#1}\mbox{}\\}
\graphicspath{{./images/}}
\titleformat{\section}{\normalfont\large\bfseries}{\thetitle. \quad }{0pt}{}[{ \titlerule[0.8pt]}]
\titleformat{\subsection}{\normalfont\bfseries}{}{}{}[]

% Portada

\begin{document}
\begin{titlepage}
    \centering
    \vfill
    \LARGE Laboratorio \#6\\
    \vskip2cm
    \large Diego Quirós Artiñano \\
    Universidad Nacional de Costa Rica \\
    EIF-202: Soporte Técnico \\ 
    Carolina Gómez Fernández \\
    15 de mayo, 2022 \\
    \vfill
    \includegraphics[width = 0.4\textwidth]{../../../UNAImage/UNA.png} \\
    \vfill
    \vfill
    % (autores separados, consultar al docente)
    % Manera oficial de colocar los autores:
    %\author{Autor(a) I, Autor(a) II, Autor(a) III, Autor(a) X}
\end{titlepage}

% Índices
\pagenumbering{roman}
    % Contenido
\addto\captionsspanish{
    \renewcommand*\contentsname{\largeÍndice}
}
\tableofcontents
\setcounter{tocdepth}{2}
\newpage
    % Figuras
% \renewcommand{\listfigurename}{\largeÍndice de fíguras}
% \listoffigures
% \newpage
%     % Tablas
% \renewcommand{\listtablename}{\largeÍndice de tablas}
% \listoftables
% \newpage

% Cuerpo
\pagenumbering{arabic}

%------------------------------------------------------------------------------------
\section*{Introducción}
\phantomsection
\addcontentsline{toc}{section}{Introducción}

%------------------------------------------------------------------------------------
\section*{Preguntas}
\phantomsection
\addcontentsline{toc}{section}{Pregutnas}

\begin{enumerate}
    \item ¿Qué es el sector de arranque o MBR? (Master Boot Record).
    

    
    \item Según la evolución del almacenamiento, explique las tecnologías
    \begin{enumerate}
        \item PMR o CMR
        \item SMR
        \item HAMR
        \item BPMR
    \end{enumerate}
    \item ¿En qué unidad se mide la velocidad de los discos duros y cuáles son las velocidades más comunes actualmente?
    \item Indique las diferencias entre SATA 1, SATA 2 y SATA 3
    \item ¿Qué es RAID (\textit{Redundant Array of Independent Disks}), es decir, "conjunto redundante de discos independientes"?. Explique los tipos de RAID:
    \begin{enumerate}
        \item RAID 0
        \item RAID 1
        \item RAID 5
        \item RAID 6
        \item RAID 10
    \end{enumerate}
    \item ¿Cómo se hace para conectar un disco maestro y otro esclavo cuando se tienen discos PATA? ¿Cómo se colocan los jumpers?
    \item Describa los pasos para particionar un disco desde Windows. Cite aplicaciones para hacer particionando
    \item ¿Cuáles son los pasos para desfragmentar un disco duro?
    \item Explique los pasos para formatear un disco
    \item Explique brevemente las siguientes características del disco
    \begin{enumerate}
        \item Tiempo medio de acceso
        \item Tiempo de lectura/escritura
        \item Tiempo medio de búsqueda
        \item Latencia Media
        \item Velocidad de rotación
        \item Tasa de transferencia
    \end{enumerate}
\end{enumerate}

%----------------------------------------------------------------------------------------------------------------------------------------------------------------------
\section*{Conclusión}
\phantomsection
\addcontentsline{toc}{section}{Conclusión}
\cite{cervantes1999}
\newpage
% Referencias
\renewcommand\refname{\large\textbf{Referencias}}
\bibliography{ref}

\end{document}